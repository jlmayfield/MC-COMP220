\documentclass[10pt]{article}
\usepackage{amsmath}
\usepackage{setspace}
\usepackage{hyperref}
\usepackage{booktabs}

\setlength{\textheight}{9in} \setlength{\topmargin}{-.5in}
\setlength{\textwidth}{6.5in} \setlength{\oddsidemargin}{0in}
\setlength{\evensidemargin}{0in}

\title{COMP220 --- Project 2}
\author{ }
\date{Fall 2016}

\begin{document}
\maketitle
\thispagestyle{empty}

\begin{abstract}
This project comes right from the text: Chapter 14, exercise 13. This document focuses on some design and implementation details and the project logistics.
\end{abstract}

\section{Project and Document Overview}

The basic requirements of the problem are well documented in the textbook and won't be repeated here. The short version is that you're implementing the core functionality of a \textit{BigInt} class for arbitrarily long unsigned integers and optionally using a simple factorial table program as a testing ground for that class.

We'll discuss algorithms for addition and multiplication in class. They are also well documented on Wikipedia and in Computer Organization and Architecture books (COMP230). This document also provides a high-level analysis for the logic of the algorithms. For full credit, your multiplication operator should be no worse than $O(n^2)$ for multiplying $n$ digit numbers and $O(n)$ or better for all other operations and methods on $n$ digit numbers.

\subsection{Project 2 Lab/Hwk Assignment}

Submit your \textit{BigInt} class header (documentation and declaration), stubs, and complete set of tests as assignment \textit{labp2} via handin. If any class code is declared and documented in a separate header than the one containing the class, then submit that header as well.

\section{Logistics}

The following are notable dates for this project:
\begin{center}
\begin{tabular}{ll}
\underline{Date} & \underline{Activity} \\
12/2 & Project 2 Lab/Hwk Due by 2pm. Some in-lab work time. \\
12/7 & Project 2 Due by End of Day
\end{tabular}
\end{center}

Your grades are based on the following criteria:
\begin{center}
\begin{tabular}{ll}
\underline{Points} & \underline{Category} \\
15  & Neatness and Good Style \\
30  & Library Design Completeness \\
30  & Library Implementation Correctness \& Efficiency \\
10 (bonus) & Factorial Table Program
\end{tabular}
\end{center}

Completeness means that all the necessary code is accounted for by way of declared and documented classes and procedures with complete test coverage. It has nothing to do with implementation or with tests passing. It's all about \textit{what} and not \textit{how}. At this point in your studies there is a very high bar for completeness.

Correctness means that your code compiles and that a set of tests with complete coverage would all pass. The main procedure should do what it's supposed to do as well.  Correctness is purely a measure of completed implementation.  If you submit code that doesn't compile, you can expect to receive no better than a D+.  This is an exception to the above rubric.

Efficiency means that multiplication of two $n$ digit numbers takes $O(n^2)$ operations and all other methods/operators require not more than $O(n)$. You should also be prepared to analyze the efficiency of these methods and operations come finals time.

\section{Base-10, Radix Number Systems, and Recursion}

Let's revisit and recall the significance of a digit's location within a number by looking at an expanded representation of the number 123.
\[
\begin{array}{rcl}
123 &=& 1*100 + 2*10 + 3*1 \\
    &=& 1*10^2 + 2*10^1 + 3*10^0
\end{array}
\]
It is the final form given above, the sum of digits multiplied by successive powers of the base of the number system (10), that clearly highlights the pattern used to represent numbers relative to a base.

In a base 10 representation, the digits come from the set $\{0,1,2,3,4,5,6,7,8,9\}$ or the integer interval $[0,10)$. Each place then represents the product of a primitive digit multiplied by a power of 10 and the total quantity is the sum of those products. We can write this pattern in a compact form using summation notation. Let $a$ be an $n$ digit number where $a_0$ is the least significant digit, $a_1$ the next, and so on to the most significant digit $a_{n-1}$. Each digit $a_i$ is drawn from the integer interval $[0,10)$. We can write $a$'s expanded form as the following sum.
\[
a = \sum\limits_{i=0}^{n-1} a_i*10^{i}
\]

Binary numbers fit the same pattern but the base is $2$ and so the digits are from $[0,2)$.
\[
a = \sum\limits_{i=0}^{n-1} a_i*2^{i}
\]
In general, for base $b$, we have digits from $[0,b)$ with representations of number $a$ of the form,
\[
\sum\limits_{i=0}^{n-1} a_i*b^{i}
\]

It's important to realize that when we use the everyday base 10 numbers we're used to to working with to write a number in a base other than 10, what we're really doing is representing a base $b$ number using representations in base $10$. This abstraction takes some getting used to. The important thing is that we have a standard way of thinking about numbers on a digit-by-digit basis that is uniform for any base $b$.

\subsection*{The Recursive Structure of Numbers}

Representing numbers as sequences of digits lends itself to recursive decomposition, which we can then use to write recursive procedures. The base case is a single digit number. For numbers with more than one digit, the least significant digit is the \textit{first} of the sequence and all the other digits the \textit{rest}. Putting these together we can write a complete recursive definition for numbers. For number $N$, written in base $b$, and with $n$ digits $d_{n-1}\ldots d_{0}$ each between $0$ and $b$,
\[
N = \left\{
\begin{array}{ll}
d_0*b^0 = d_0  & n = 1 \\ \\
(d_{0}*b^{0}) + \left(\sum\limits_{i=1}^{n-1}d_{i}b^{i}\right) & n > 1 \\
\end{array}
\right.
\]
In the recursive case, parenthesis were used to separate the singular element, the \textit{first}, from the recursive part, the \textit{rest}.

We want the rest of an $n$ digit number to be an $n-1$ digit number. The recursive decomposition we gave above implies another $n$ digit number with a $0$ is the one's place. For the number $123$, we'd want our sum to express $12$ not $120$. We know that in context, it's $120$, but in terms of the recursive decomposition it should be a two digit number. This can be accomplished by factoring out 10 from our previous recursive form.
\[
\left(\sum\limits_{i=1}^{n-1}d_{i}b^{i}\right) = \left(\sum\limits_{i=1}^{n-1}d_{i}b^{i-1}\right)b
\]
Pay extra attention to the different formulations of the rest, they'll be important later when we formulate a recursive implementation of multiplication.

Recognizing and understanding this recursive structure is important because it allows you to lay out a basic recursive template for digit by digit operations on numbers. If you need a refresher, that template is:
\begin{verbatim}
if( base-case ){
  operate on base element as/if needed
}
else{
  deconstruct into first and rest
  operate on pieces as/if needed
  recombine results of by-parts computation as/if needed
}
\end{verbatim}
This pattern should be very, very, very familiar from COMP160. We'll now apply it towards the problem of addition and multiplication.

\subsection*{Iterative Addition}

We can map our recursive structure onto the problem of addition without much trouble:
\[
\begin{array}{rcl}
A+B &=& \left(A_0 + \left( \sum\limits_{i=1}^{n-1} A_i10^{i-1} \right)10  \right) +
\left(B_0 + \left( \sum\limits_{i=1}^{n-1} B_i10^{i-1} \right)10  \right) \\ \\

&=& (A_0 +  B_0) + \left( \sum\limits_{i=1}^{n-1} A_i10^{i-1} +
\sum\limits_{i=1}^{n-1} B_i10^{i-1} \right)10
\end{array}
\]
It's a short hop from this expression to a recursive procedure that effectively traverses to $n$ digit numbers in parallel. This is not, however, what we'll do.  Instead we'll implement a more familiar, iterative process for addition.

Before delving into $n$ digit addition, let's' consider the base case, where $A$ and $B$ are single digit numbers. Intuitively, you know what happens. We add two digits and we get either another single digit number ($1+2=3$) or a two digit number ($5+7=12$). Another way of saying this is that the addition of two single digit numbers always produces a single digit sum and a single digit carry where the carry is the excess of the actual sum equal to or above the base. For $1+2$, the sum is $3$ and the carry is $0$. For $5+7$ the sum is $2$ and the carry is $10 = 1 * 10^1$.

The carry can be uniformly extracted from the complete sum by strict integer division by the base, 10. When the carry is $0$ then the sum must be in $(10,0]$ and dividing by 10 gives 0. When the carry isn't $0$ then the sum is in $(20,10]$ and dividing by 10 gives the ten's digit, 1. To avoid any confusion, we'll express strict integer division, like you see in C++, mathematically as the \textit{floor function}, i.e.\ division rounded down to the nearest integer. For integers $x$ and $y$ we see the follow C++ and mathematical expressions for what we know as integer division.
\[
\begin{array}{cc}
\mbox{\underline{C++}} & \mbox{\underline{Mathematics}} \\ \\
x/y & \left\lfloor\dfrac{x}{y}\right\rfloor
\end{array}
\]

By similar logic, we can extract the single digit sum by using the modulo operator with the base 10. Taking any number from $(100,0]$ modulo 10 will give you exactly the one's place of that number. In mathematics we just write $\mod$ where you use $\%$ in C++. When dealing with strictly positive integers, as we are, the two operations are equivalent. They do differ when dealing with negative values though. For positive integers $x$ and $y$ we see the follow C++ and mathematical expressions for the modulo operation.
\[
\begin{array}{cc}
\mbox{\underline{C++}} & \mbox{\underline{Mathematics}} \\ \\
x\%y & x \mod y
\end{array}
\]

Finally, it's worth noting that when adding only single digit, positive numbers, the total sum will be in $(100,0]$ as long as we're adding together not more than 11 numbers. (Do you see why?) This means simple division and modulo can be used to extract the single digit sum and carry for the sum.

Now, let's restate the addition base case in mathematical notation. For single digit numbers $A$ and $B$,
\[
\begin{array}{rcl}
A+B &=& \left\lfloor\dfrac{A+B}{10}\right\rfloor10^1 + ((A+B)\mod 10)10^0 \\ \\
&=& \left\lfloor\dfrac{A+B}{10}\right\rfloor*10 + ((A+B)\mod 10)
\end{array}
\]
where the first term is the carry and the second term is the sum.

The $n$ digit algorithm utilizes the basic logic of a carry-sum adder. It's logic is iterative and accumulates an $n$ digit sum and a carry. While we consider single digit numbers the base case, it's also reasonable to say that a zero digit number is $0$ and that $0+0$ has a sum and carry of $0$.
\[
\begin{array}{rcl}
S_{0} &=& 0 \\
c_{0} &=& 0 \\
\end{array}
\]
With the initial values of the accumulators set, we can now set down iterative accumulation logic. Assume we've traversed and summed across the first $k < n$ digits of the $n$ digit numbers $A$ and $B$. We should have accumulated a $k$ digit sum $S_k$ and a single carry digit $c_k$. The iterative logic then accumulates the ${(k+1)}^{th}$ digits of $A$ and $B$ by using them to compute the ${(k+1)}^{th}$ digit of the sum $S_{k+1}$ and the carry $c_{k+1}$ used for the next digit.
\[
\begin{array}{rcl}
S_{k+1} &=& \left\lfloor \dfrac{A_{k+1} + B_{k+1} + c_k}{10} \right\rfloor\times10^k + S_k \\ \\
c_{k+1} &=& (A_{k+1} + B_{k+1} + c_k) \mod 10 \\
\end{array}
\]
Notice that in practice we should only need to compute the value of the $k+1$ digit and append that to the previous sum $S_k$ (Do you see why?).

If begin with the initial sum and carry of $0$ and repeat across the digits in least to greatest order then we should get the final sum as $c_n10^{n+1}+S_n$ (notice we need to incorporate the final carry after accumulating the $n$ digit sum). You should recognize this as how you were taught to do addition back in grade school. From here you should be able to adjust for the case when the two numbers are not of equal length but first reducing it to the sum of two equal length numbers then adjusting for any digits left off of that sum.

\subsection*{Recursive Multiplication}

When you learned to do multiplication by hand you probably started by learning the multiplication table for single digit numbers. You should now recognize this as memorizing the base case for single digits. The only thing we'll say now about this base case is that for any two single digit numbers, multiplication produces at most a two digit number (Do you see why?). This means we can use the same division and modulo technique for extracting the first and second digits from single digit multiplication.

The algorithm you learned to go with this base case required that you multiply the first number by each digit of the second number and shift the result such that the result started in the same place as the digit from the second number. You then add all these results. In case you need a reminder, here's an example:
\[
\begin{array}{lllll}
 & &1&2&3 \\
\times & & &1&5 \\ \midrule
 & &6 &1 &5 \\
+ &1 &2 &3 &0 \\ \midrule
 &1 &8 &4 &5
\end{array}
\]

This process is very cleanly expressed and formalized using our summation-based notation. From there we can reformulate it around the recursive decomposition of $B$. For $n$ digit number $B$ and any number $A$,
\[
\begin{array}{rcl}
A \times B &=& A \times \left(\sum\limits_{i=0}^{n-1}B_i10^i \right)\\ \\
 &=& \sum\limits_{i=0}^{n-1}AB_i10^i \\ \\
 &=& AB_0 + \sum\limits_{i=1}^{n-1}AB_i10^i \\ \\
 &=& AB_0 + \left(\sum\limits_{i=1}^{n-1}AB_i10^{i-1}\right)10 \\ \\
\end{array}
\]
That algorithm you were given, in fact, asked you to distribute the first operand, $A$, across the expanded representation of the second, $B$, and then carry out the computation implied by the expanded representation of the result. These implied computations were simpler: multiplication of an $n$ digit number by a single digit, multiplication by a power of the base, and basic sums. As the final line of the equation show, this cleanly maps to the recursive structure of $B$. So, we know how to do the addition, but we need to work out the details of the more constrained variants of multiplication.

The first and simplest variant of multiplication is that of a number $A$ by a power of the base. For power $10^i$, this amounts to a shift to the left by $i$ places, padding with $0$ as you shift. Mapping this to our our summation-based notation clearly illustrates why. For $n$ digit number $A$,
\[
\begin{array}{rcl}
A \times 10^k &=& \left( \sum\limits_{i=0}^{n-1}A_i10^i \right) \times 10^k \\ \\
 &=& \sum\limits_{i=0}^{n-1}A_i10^i10^k \\ \\
 &=& \sum\limits_{i=0}^{n-1}A_i10^{i+k}
\end{array}
\]
In the case of your BigInt class, no multiplication need actually take place here, we simply need to extend the length of the number by adding the appropriate number of zeros in the low order places.

The other helper we need carries multiplication of an $n$ digit number by a single digit number.  This is really just a special case of the general multiplication process and we can see this by mapping it out across the summation expansion of the $n$ digit number. Let $b$ be a single digit number with $A$ an $n$ digit number. Then,
\[
\begin{array}{rcl}
A \times b &=& \left( \sum\limits_{i=0}^{n-1}A_i10^i \right) \times b \\ \\
 &=& \sum\limits_{i=0}^{n-1}A_{i}b10^i  \\ \\
 &=& A_0b + \sum\limits_{i=1}^{n-1}A_{i}b10^i \\  \\
 &=& A_0b + \left(\sum\limits_{i=1}^{n-1}A_{i}b10^{i-1}\right)10 \\
\end{array}
\]
For the sake of efficiency, it's important to note that the term $A_0b$ is at most two digits. If our general purpose addition operation won't stop after dealing with only those digits and their potential carry, then we should work out this operation to do so. What needs to be avoided is a complete traversal of the other $n-2$ or $n-3$ digits of the final product.

We can now put the general multiplication algorithm together. Let's revisit the recursive formulation:
\[
\begin{array}{rcl}
A \times B &=& AB_0 + \left(\sum\limits_{i=1}^{n-1}AB_i10^{i-1}\right)10 \\ \\
\end{array}
\]
Implied by this formulation are calls to the the helpers and you should take the time to translate this notation into a set of nested or sequenced procedure calls. To help towards this end, reconsider $123 \times 15$:
\[
\begin{array}{rcl}
123 \times 15 &=& 123\times 5 + (123\times 1)\times 10 \\
 &=& 615 + 1230 \\
 &=& 1845
\end{array}
\]
Finally, let's look at something with a bit more recursive depth $123 \times 115$.
\[
\begin{array}{rcl}
123 \times 115 &=& 123\times 5 + (123\times 11)\times 10 \\
 &=& 615 + 1353 \times 10 \\
 &=& 615 + 13530 \\
 &=& 14145
\end{array}
\]
If you're not seeing the recursive patterns at play both in the structure of the numbers and in the process of multiplication, you should continue to work examples by hand in the same fashion as the past two examples. Doing the same thing for the helpers is also highly recommended as it lets you peel back the curtain on this top-level perspective.

\subsection*{Mathematical Decomposition}

Your \textit{BigInt} class embeds the natural recursive structure of a number in an explicit linked-list structure. You can recursively process the number by recursively processing the list and will do so when implementing addition and multiplication. However, it's important to know how to deconstruct integer values in a digit-by-digit fashion as well.

You've already seen the underlying mechanism by which you can break down a number into its digits: integer division and modulo by the number's base. If you'd like to work from least to most significant digit, then a successive divisions and modulo by 10 gets the job done. The first digit is extracted with $\mod$ and the rest with division by 10.
\[
\begin{array}{l}
1234 \mod 10 = 4 \\ \\
\left\lfloor \dfrac{1234}{10} \right\rfloor = 123
\end{array}
\]
Repeating this iteratively or recursively gives you the digits in least to greatest order.

Let's now visit the question of the number of digits in the base 10 representation of a number. If we knew this ahead of time we could count out the number of divisions by 10 it would take to reach the base case, a single digit or 0. Let's start with an easy case where the number is exactly a power of 10. This problem takes on a familiar form as the final base case is exactly $1$:  ``how many times do we need to divide by 10 to get 1?'' Logarithms come to the rescue. For number $N = 10^k$,
\[
\begin{array}{rcl}
\dfrac{N}{10^k} &=& 1 \\ \\
N &=& 10^k \\
\log_{10} N k
\end{array}
\]
So, it takes $k$ division to get to the last digit and we need $(\log_{10} N) + 1$ digits to represent $N = 10^k$.  For example, $10^2 = 100$ requires 3 digits, $1 = 10^0$ requires 1 digit, $10000 = 10^4$ requires 5 digits, and so forth.

If a number isn't an exact power of 10, like $123$ then it must fall between two powers of 10, like $100$ and $1000$, and the number of digits must be the same as the number needed for the power of 10 below the number in question. Mathematically, there's two ways to get this number, one involving rounding down and one involving rounding up, i.e.\ the ceiling function.
\[
\lfloor \log_{10} N \rfloor + 1  = \lceil \log_{10} N \rceil
\]
Using the floor gives us something consistent with what happens for exact powers of 10. Using the ceiling, saves us the $+1$ by recognizing that rounding up is the same as rounding down and then adding one. Either way you view it, we now know how many digits we need to represent the number $N$ is base 10: $\lfloor\log_{10}(N)\rfloor + 1$

Finally, it's important to notice that changing the base to $b$ from $10$ simply means replacing all the 10 and powers of 10 logic with $b$ and powers of $b$. If you do all of this with a base 10 system, then you're just expressing the value of each base $b$ digit as a base $10$ number.

\section{The Path to the Private Details of a Class}

The \textit{BigInt} class represents numbers using a general list-like structure. As opposed to previously encountered number classes like Rational and Complex, this class has a non-obvious representation. The clients of our class need not worry about this nor should they have to change how they think about numbers because of how we implemented these particular numbers. The challenge is then to provide a natural public interface while efficiently handling the private representation. The single most important thing to keep in mind is that the public interface to your class should not in any way be dependent on the underlying implementation. If you change the implementation, the interface shouldn't change as a result.

One problem we face is that we want keep operators like $+$ and $*$ outside the class but their implementation clearly needs access to the private structure. The solution you should use is a common design idiom. The non-class operator/procedure makes a call to a public class method and the public class method works with the private implementation. In the case of $operator+$, we can call to a BitInt method named $add$ that actually carries out the addition. So, when the programmer invokes the operator with $a + b$, the operator in fact makes a call to $a.add(b)$ which then can work with the underlying implementation of $a$ and $b$ directly.  The alternative, which I do not want you to use, is making \textit{operator$+$} a \textit{friend} procedure like you saw in the textbook's implementation of the \textit{Rational} class. This gives the operator access to the private parts of the BigInt class.

The algorithms for addition and multiplication aren't easily or efficiently expressed in terms of the expected public class methods. More to the point, a publicly accessible \textit{first} and \textit{rest} for recursive decomposition of a number is not really a standard, expected interface for numbers. Even if it were, we'll need to be careful about copying issues.  Is the \textit{rest} a copy or an alias? If it's the later, should it be read-only or mutable? It's easy to get this wrong, so we'll just avoid it altogether.

This means we need a series of private methods to carry out underlying recursive and iterative algorithms for the list representation of the number. You have two paths towards this end. First, make private methods true class methods.  To facilitate this it is often helpful to have private constructor that take in instance of the private data type to initialize the object. Users should almost never directly initialize private data, but you, the class designer and implementer, might have need to in the context of other class implementation. For \textit{BigInt} this would mean a private constructor that takes a pointer to a list node. Such a constructor can turn the \textit{rest} of a number into a \textit{BigInt}. In addition to constructors, you can make private methods like \textit{first} and \textit{rest} that let you easily work with a \textbf{BigInt} in a recursive fashion. This \textit{rest} could, in fact, be an alias to the rest of another list and by making it private we can insure not unexpected mutation occurs while working with the alias.

The second option is to drop all the class stuff and write good old fashioned procedures for the underlying private list structure. By making these procedures a part of the class definition, you can easily call them in the context of public and private class methods. For example, let's say you have some class \textit{Fee} with a private integer \textit{anum} and string \textit{astr} and you need to do some involved, perhaps recursive, process involving \textit{anum} and \textit{astr} that results in an integer. That process can be defined as a procedure \textit{foo} taking an int and a string and returning an int. It doesn't need to be publicly accessible because it's only meant to facilitate the class implementation. The private part of the class is a perfect place for it. However, it's not a class method, it's just a traditional procedure. To make a private, pure procedure inside a class you declare the procedure using the \textit{static} keyword like so:
\begin{verbatim}
// inside the Fee class ...
   private:
      int anum;
      std::string astr;

      // static, private procedure.. NOT A CLASS METHOD.. NO THIS
      static int foo(int x, std::string str);
\end{verbatim}

The static procedure \textit{foo} isn't a class method. It has no \textit{this} and therefore no implicit \textit{anum} and \textit{astr}.  As a private procedure it can only be called within the \textit{Fee} implementation. The call to a static class procedure is different than class methods and non-class procedures. The class name, \textit{Fee} in this case, acts like a namespace name. Here we see a call to \textit{foo} from within some Fee class method.
\begin{verbatim}
//inside a (possibly Public) class method for Fee

 ... Fee::foo(this->anum,this->astr) ...
\end{verbatim}
Fee is not a namespace. You \textit{must} always include the \textit{Fee::} in front of \textit{foo}. There is no shortcut around this. Now, because this procedure call occurs within a class method, we can use that method's \textit{this} pointer to access and pass the private data stored in \textit{anum} and \textit{astr}, and because \textit{foo} is private to the class it's scope is contained to exactly the place it will be used, the \textit{Fee} implementation. In effect, the object hands off the private data to a regular old procedure that can only be called internally by instance's of that object's type. This is exactly how we can keep special purpose helper procedures out of the public/global scope.

You are free to use either private class methods or private static procedures. The former has the advantage of being consistent with the public facing part of the class. Everything in the class definition will be methods or variables. The later forces you to mix pure procedures with class methods within a class and this can require careful attention. Using a call to \textit{this} within a static method will cause compiler errors. On the other hand, you have a lot of practice with pure procedures and they aren't complicated by implicit parameters like the \textit{this} object pointer. Which route you take is up to you. Just be careful when pointers get involved and know when you're working with a shallow-copy (an aliased structure) or deep-copies (an equivalent structure).

\subsection*{Testing Private Methods}

By definition, private methods are not accessible outside of the class. This means they are not accessible where we do our testing. One way around this is to not test them directly but instead carefully and deliberately test the public methods and procedures that use them. It's OK if you go this route, but when those top-level tests fail and you're not sure why, it's nice to be able to test the private code.

The gTest framework discusses some strategies for testing private class methods in the documentation:
\begin{quote}
\url{https://code.google.com/p/googletest/wiki/AdvancedGuide#Testing_Private_Code}.
\end{quote}
The simplest route is using the Friend Tests.  The problem with this technique is you now have testing code where we don't typically want it, in the class declarations and definitions proper. For the purpose of this project, that's OK\@. Another simple approach is to debug the private code by outputting the values of your variables as the code executes to see what's off. This is a very traditional approach and requires no non-standard libraries.  It does force you to go back and cleanup a bunch of output statements after you're done debugging though.

\section{Bonus Program}

For some bonus points you can write a program that uses your big int library to print out a nice table of the first $n$ factorials for any positive $n$ given at the command-line. For example, if 5 were passed for $n$ it should print:

\begin{tabular}{rcr}
  n & & n! \\
  1 & & 1 \\
  2 & & 2 \\
  3 & & 6 \\
  4 & & 24 \\
  5 & & 120
\end{tabular}


\end{document}
