\documentclass[nobib]{tufte-handout}
\usepackage{amsmath}
%\usepackage{setspace}
\usepackage{hyperref}
\usepackage{booktabs}

%\setlength{\textheight}{9in} \setlength{\topmargin}{-.5in}
%\setlength{\textwidth}{6.5in} \setlength{\oddsidemargin}{0in}
%\setlength{\evensidemargin}{0in}

\title{Standford Library to Standard Library Cheat sheet \\ COMP220 Data Structures}
\author{}
\date{Fall 2016}

\begin{document}
\maketitle

\thispagestyle{empty}

\begin{abstract}
  This document list standard library equivalents to the collections classes in the Stanford C++ library used in the text book.  Handling of special cases and errors may differ between the Stanford and Standard equivalents presented here. You should verify the specifications with the appropriate resource.
\end{abstract}

\subsection*{Vector}

The C++ std::vector class\sidenote{\url{http://www.cplusplus.com/reference/vector/vector/}} can be used in place of the Stanford C++ Vector class\sidenote{pg 197}.

\begin{figure}[!htbp]
\begin{tabular}{lll}
  \underline{Operation} & \underline{Stanford} & \underline{Std} \\
  Empty Constructor & Vector$<$t$>$() & std::vector$<$t$>$() \\ %chktex 36
  Fill Constructor & Vector$<$t$>$(size,val) & std::vector$<$t$>$(size,val) \\  %chktex 36
  Size Query Method & v.size() & v.size() \\  %chktex 36
  Empty Predicate & v.isEmpty() & v.empty() \\  %chktex 36
  Element Selector & v.get(idx) & v.at(idx) \\  %chktex 36
                   & v[idx] & v[idx] \\  %chktex 36
  Element Mutator & v.set(idx,val) & v[idx] = val \\ %chktex 36
  Add to Back & v.add(val) & v.push\_back(val) \\ %chktex 36
  Insert at & v.insertAt(idx,val) & v.insert(std::begin(v) + idx, val) \\ %chktex 36
  Remove at & v.removeAt(idx) & v.erase(std::begin(v) + idx) \\ %chktex 36
  Erase all & v.clear() & v.clear() \\ %chktex 36
  Concatenate & v + w & v.insert(std::end(v),std::begin(w),std::end(w)) \\ %chktex 36
  Repeated Add & v += a,b,c,d,e \ldots & N.A. \\ %chktex 36
\end{tabular}
\caption{Vector to std::vector}
\end{figure}

\subsection*{Stack}

The std::stack class\sidenote{\url{http://www.cplusplus.com/reference/stack/stack/}} can be used in place of the Stanford Stack class\sidenote{pg 211}.

\begin{figure}[!htbp]
\begin{tabular}{lll}
  \underline{Operation} & \underline{Stanford} & \underline{Std} \\
  Empty Constructor & Stack$<$t$>$() & std::stack$<$t$>$() \\ %chktex 36
  Size Query & s.size() & s.size() \\ %chktex 36
  Empty Predicate & s.isEmpty() & s.empty() \\ %chktex 36
  Push Element & s.push(val) & s.push(val) \\ %chktex 36
  Pop Element & s.pop() & s.pop() \\ %chktex 36
  Peek at Top & s.peek() & s.top() \\ %chktex 36
  Remove all Elements & s.clear() & N.A. \\ %chktex 36
\end{tabular}
\caption{Stack to std::stack}
\end{figure}

\newpage \thispagestyle{empty}

\subsection*{Queue}

The std::queue class\sidenote{\url{http://www.cplusplus.com/reference/queue/queue/}} can be used in place of the Stanford C++ Queue class\sidenote{pg 217}.

\begin{figure}[!htbp]
\begin{tabular}{lll}
  \underline{Operation} & \underline{Stanford} & \underline{Std} \\
  Empty Constructor & Queue$<$t$>$() & std::queue$<$t$>$() \\ %chktex 36
  Size Query & q.size() & q.size() \\ %chktex 36
  Empty Predicate & q.isEmpty() & q.empty() \\ %chktex 36
  Enqueue Element & q.enqueue(val) & s.push(val) \\ %chktex 36
  Dequeue Element & s.dequeue() & s.pop() \\ %chktex 36
  Peek at Front & s.peek() & s.front() \\ %chktex 36
  Remove all Elements & s.clear() & N.A. \\ %chktex 36
\end{tabular}
\caption{Queue to std::queue}
\end{figure}

\subsection*{Map}

The std::map class\sidenote{\url{http://www.cplusplus.com/reference/map/map/}} can be used in place of the Stanford C++ Map class\sidenote{pg 226}.

\begin{figure}[!htbp]
\begin{tabular}{lll}
  \underline{Operation} & \underline{Stanford} & \underline{Std} \\
  Empty Constructor & Map$<$kt,vt$>$() & std::map$<$kt,mt$>$() \\ %chktex 36
  Size Query & m.size() & m.size() \\ %chktex 36
  Empty Predicate & m.isEmpty() & m.empty() \\ %chktex 36
  Select Element & q.get(key) & m.at(key) \\ %chktex 36
   & m[key] & m[key] \\
  Add Element & m.put(key,val) &  m.insert(std::pair$<$kt,vt$>$(key,val))  \\ %chktex 36
    & m[key] = val & m[key] = val \\
  Remove Key+Value & m.remove(key) & m.erase(key) \\ %chktex 36
  Key Containment Predicate & m.containsKey(key) & m.count(key) \\ %chktex 36
  Remove all Elements & m.clear() & m.clear() \\ %chktex 36
\end{tabular}
\caption{Map to std::map}
\end{figure}

Alternatively, one can use std::unordered\_map\sidenote{\url{http://www.cplusplus.com/reference/unordered_map/unordered_map/}} to improve the performance of certain map operations.


\newpage \thispagestyle{empty}

\subsection*{Set}

The std::set class\sidenote{\url{http://www.cplusplus.com/reference/set/set/}} can be used in place of the Stanford C++ Set class\sidenote{pg 232}. The std::includes, std::set\_intersection, std::set\_difference, and std::set\_union functions are found in the \textit{algorithm} library\sidenote{\url{http://www.cplusplus.com/reference/algorithm/}}

\begin{figure}[!htbp]
\begin{tabular}{lll}
  \underline{Operation} & \underline{Stanford} & \underline{Std} \\
  Empty Constructor & Set$<$t$>$() & std::set$<$t$>$() \\ %chktex 36
  Size Query & s.size() & s.size() \\ %chktex 36
  Empty Predicate & s.isEmpty() & s.empty() \\ %chktex 36
  Add Element & s.add(val) & s.insert(val) \\ %chktex 36
  Remove Element & s.remove(key) & s.erase(key) \\ %chktex 36
  Containment Predicate & s.contains(val) & s.count(key) \\ %chktex 36
  Remove all Elements & s.clear() & s.clear() \\ %chktex 36
  Is Subset of & s.isSubsetOf(r) & std::includes(std::begin(s),std::end(s), \\ %chktex 36
   & & std::begin(r),std::begin(s))   \\ %chktex 36
  Get First & s.first() & *(std::begin(s)) \\ %chktex 36
  Union & s + r & std::union(std::begin(s),std::end(s), \\ %chktex 36
  & & std::begin(r),std::end(r),std::begin(result\_set)) \\ %chktex 36

  Intersection & s * r & std::set\_intersection(std::begin(s),std::end(s), \\ %chktex 36
  & & std::begin(r),std::end(r),std::begin(result\_set)) \\ %chktex 36
  Difference & s $-$ r & std::set\_difference(std::begin(s),std::end(s), \\ %chktex 36
  & & std::begin(r),std::end(r),std::begin(result\_set))\\ %chktex 36
\end{tabular}
\caption{Set to std::set}
\end{figure}

Alternatively, one can use std::unordered\_set\sidenote{\url{http://www.cplusplus.com/reference/unordered_set/unordered_set/}} to improve the performance of certain set operations.


\end{document}
