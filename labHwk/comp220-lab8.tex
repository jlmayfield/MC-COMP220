\documentclass[10pt]{article}
\usepackage{amsmath}
\usepackage{setspace}
\usepackage{hyperref}
\usepackage{listings}
\usepackage{color}

\definecolor{dkgreen}{rgb}{0,0.6,0}
\definecolor{gray}{rgb}{0.5,0.5,0.5}
\definecolor{mauve}{rgb}{0.58,0,0.82}

\lstset{
  %frame=tb,
  language=C++,
  aboveskip=3mm,
  belowskip=3mm,
  showstringspaces=false,
  columns=flexible,
  basicstyle={\small\ttfamily},
  numbers=none,
  numberstyle=\tiny\color{gray},
  keywordstyle=\color{blue},
  commentstyle=\color{dkgreen},
  stringstyle=\color{mauve},
  breaklines=true,
  breakatwhitespace=true,
  tabsize=3
}



\setlength{\textheight}{9in} \setlength{\topmargin}{-.5in}
\setlength{\textwidth}{6.5in} \setlength{\oddsidemargin}{0in}
\setlength{\evensidemargin}{0in}

\title{COMP220 --- Lab 8}
\author{ }
\date{Fall 2016}

\begin{document}
\maketitle
\thispagestyle{empty}

\begin{abstract}
For this lab you'll write a simple array comparison procedure that is necessary for testing array equalitys. 
\end{abstract}


\section*{Array Testing}

Using gtest to check if two arrays contain the same contents is complicated by the $==$ operator will return true for array comparison if and only if the two array variables have the same address stored in them. Put another way, two arrays are eqaul (according to operator==) if and only if they have the same starting address. This means you'll need to do one of two things:
\begin{enumerate}
\item Loop through the expected and actual array and compare each individual element.
\item Write a helper procedure to test if two arrays have the same contents. 
\end{enumerate}

The later option is more reusable and something we'll tackle today. 

\subsection*{arrayEquals}

Design and implement a procedure named \textit{arrayEquals} that takes two arrays of integers and returns true if they have the same contents and false otherwise. When passing arrays we must pass the array variable and the array size.  Thus your procedure has the following signature:

\begin{lstlisting}
bool arrayEquals(int lhs[], int lsize, int rhs[], int rsize);
\end{lstlisting}

Write a library containing this method. Submit the documentation, declaration, a stub, and gTests as assignment \textit{lab8} and the completed implementation as assignment \textit{hwk6}

\end{document}
