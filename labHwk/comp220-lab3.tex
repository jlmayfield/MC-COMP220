\documentclass[10pt]{article}
\usepackage{amsmath}
\usepackage{setspace}
\usepackage{hyperref}

\setlength{\textheight}{9in} \setlength{\topmargin}{-.5in}
\setlength{\textwidth}{6.5in} \setlength{\oddsidemargin}{0in}
\setlength{\evensidemargin}{0in}

\title{COMP220 - Homework 3 \& Lab 3}
\author{ }
\date{Fall 2015}

\begin{document}
\maketitle

\begin{abstract}

\end{abstract}

\section{Homework 3}

\begin{center}
\textbf{\textit{Due at the start of lab on 9/10}}
\end{center}

For homework you must setup a project to work with both Google Tests and the StanfordCPP Library. Setting up the Stanford library is described here \url{https://jlmayfield.github.io/MC-COMP220/general/comp220-cbAndscppl.pdf}. To test your setup incorporate the files found in \textit{c220-hwk3.zip}, which you'll find in the course home directory (/home/comp220/fa15). The file \textit{HelloWorld.cpp} is a basic main program that utilizes the \textit{random.h} library from the StanfordCPP library. It should be compiled by your Debug and Release builds.  The file \textit{HelloTests.cpp} contains a set of Google Tests that utilize the StanfordCPP \textit{vector.h} and \textit{random.h} library. If your project is setup correctly in C::B, then you should be able compile and run the Debug/Release executable and the Test executable produced by this code. 

\textit{I will come around at the start of lab to check that you have a project setup that can compile and run these two programs.}

\section{Lab 3}

For lab you're to work on some vector problems from the book. Vectors should be familiar to you from COMP161 so we'll just dive right in and adapt to the StanfordCPP vector. Choose one of the following to work on (even better, just do them all). 
\begin{enumerate}
\item \textit{Chapter 5, Exercise 1} This is great practice at designing Input procedures, which isn't something we've reviewed yet. 
\item \textit{Chapter 5, Exercises 2-3} Some pretty standard functions. These should let you practice basic iterative problem solving and the \textit{fold} pattern.
\item \textit{Chapter 5, Exercise 4} Design this one as an Output procedure. This would be good practice for tomorrows quiz.
\end{enumerate}
We'll likely discuss some alternative design considerations in class. For now, just go at the problems as they're laid out in the book.

\end{document}