\documentclass[10pt]{article}
\usepackage{amsmath}
\usepackage{setspace}
\usepackage{hyperref}

\setlength{\textheight}{9in} \setlength{\topmargin}{-.5in}
\setlength{\textwidth}{6.5in} \setlength{\oddsidemargin}{0in}
\setlength{\evensidemargin}{0in}

\title{COMP220 --- Homework 3 \& Lab 3}
\author{ }
\date{Fall 2016}

\begin{document}
\maketitle

\begin{abstract}

\end{abstract}

\section{Homework 3}

\begin{center}
\textbf{\textit{Due at the start of lab on 9/8}}
\end{center}

For homework you need to get your Code::Blocks project setup, stub the procedures described below, and write tests for the procedures as well.  In short, get all the design prep work done ahead of time so that you can focus on the programming in lab. \textit{I will come around at the start of lab to check that you have a project setup and that you can compile and run the tests.}

\section{Lab 3}

This lab is an extension of exercise 11 in chapter 7 of the text.  You need to write a suite of functions for converting integers to strings and vice versa.  The function \textit{intToStr} would take the integer 137 and produce the string ``137''. Conversely, the function \textit{strToInt} would take ``-256'' and produce the integer $-256$. You should develop a recursive and iterative implementation of each function. Use C++ namespaces to distinguish the implementations rather than changing the names of the functions as given above.


Exercise 7 briefly discusses the recursive decomposition of integers in terms of the least significant digit and all other digits. Mathematically, for positive integers we can express this as follows:
\begin{equation}
  x = 10 \left\lfloor \frac{x}{10} \right\rfloor + (x \pmod{10})
\end{equation}

At the end of lab, submit your source code via handin as assignment lab3.

\end{document}
