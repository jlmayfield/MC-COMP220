\documentclass[10pt]{article}
\usepackage{amsmath}
\usepackage{setspace}
\usepackage{hyperref}

\setlength{\textheight}{9in} \setlength{\topmargin}{-.5in}
\setlength{\textwidth}{6.5in} \setlength{\oddsidemargin}{0in}
\setlength{\evensidemargin}{0in}

\title{COMP220 --- Lab 7}
\author{ }
\date{Fall 2016}

\begin{document}
\maketitle

\begin{abstract}
This is a choose your own adventure lab.  You must simply make some constructive progress on the problem set that will feed next week's exam. At the end of lab, submit any code you've written as assignment \textit{lab7}.
\end{abstract}

\section{The Problem Set}

The following problems will put you through your paces on the collections discussed in chapter five: 12, 13, 14, 15, 19, 20, 21, 22, 23, 24, and 25.  As we discussed in class, you should approach these using our program design strategies by decomposing each problem into basic procedures for which you write documentation, declaration, stubs, tests, and finally an implementation. When applicable, consider both solutions involving both functions and mutators.

\end{document}
