\documentclass[10pt]{article}
\usepackage{amsmath}
\usepackage{setspace}
\usepackage{hyperref}

\setlength{\textheight}{9in} \setlength{\topmargin}{-.5in}
\setlength{\textwidth}{6.5in} \setlength{\oddsidemargin}{0in}
\setlength{\evensidemargin}{0in}

\title{COMP220 --- Lab 2 \& Homework 2}
\author{ }
\date{Fall 2015}

\begin{document}
\maketitle

\begin{abstract}
  For your second lab and homework assignment you'll be practicing the analysis of recursion and the translation from simple recursion to iteration and back.
\end{abstract}

\section{Lab 2}

In section 7.4 of your text two versions of a procedure to detect palindromes are given to you. The first suffers from some common inefficient pitfalls which the second avoids.  You are two do the following:
\begin{enumerate}
  \item Translate the more efficient of the two versions to an iterative, loop-based procedure.
  \item Produce recurrence relations for both versions. Solve those recurrence relations using the unrolling technique. Sketch out the basis for an inductive proof of the correctness of your solution. This work can either be done by hand and submitted directly to the instructor or you may type it up and submit it along with the remainder of the lab.

\end{enumerate}

\section{Homework 2}

\begin{center}
  \textbf{Submit source code as assignment hwk2 via handin prior to the next lab.}
\end{center}

Section 7.5 of your text provides a recursive implementation of Binary Search. Your job is to:
\begin{enumerate}
  \item Translate this to an iterative, loop-based procedure. Be sure to properly document, declare, and test this code. Iterative versions of this algorithm are surely found all over the internet, try to do this on your own.
  \item Write out the recurrence relation for this code. Place it in a comment block with your iterative code.
  \item Solve for the closed form of the recurrence using the unrolling technique. (Optional) Use a proof by induction to verify your closed form solution.
\end{enumerate}

\end{document}
