\documentclass[10pt]{article}
\usepackage{amsmath}
\usepackage{setspace}
\usepackage{hyperref}

\setlength{\textheight}{9in} \setlength{\topmargin}{-.5in}
\setlength{\textwidth}{6.5in} \setlength{\oddsidemargin}{0in}
\setlength{\evensidemargin}{0in}

\title{COMP220 - Lab 2 \& Homework 2}
\author{ }
\date{Fall 2015}

\begin{document}
\maketitle

\begin{abstract}
This lab is essentially more review. This time you'll need to do a standard function as well as an I/O function. Chapter 4, section 4 of the text provides a good overview of the streaming I/O class hierarchy. The chapter also discusses formatted output in section 1. Lecture notes 9 from COMP161 discuss the application of our design process to I/O procedures.  This includes how to use gTest and string streams to write unit tests for I/O based procedures. 
\end{abstract}

\section{Lab 2}

Your task is a slightly modified version of exercise 2 from chapter 4 of the text. For lab you should:
\begin{itemize}
\item Document, declare, stub, and write tests for the \textit{windChill} function as described in chapter 2 exercise 4.
\item Document, declare, stub, and write a test for a procedure to print the wind chill table as discussed in chapter 4 exercise 2. Your table does not need the graphics and can essentially be just the data will row and column headers (i.e. just the numbers). 
\item Write a simple main procedure which calls your table printing procedure. (Note that this should compile and run but will only do what your stubs indicate it should do).
\end{itemize}
Your functions can both be in a single library and namespace. Once these tasks are done, or lab time is up, submit your source code as \textit{lab2} using the handin script. 

\section{Homework 2}

For homework you should complete the implementation of both functions and debug as necessary. Submit the source code for the completed program as \textit{hwk2} using handin. This assignment is due by \textbf{8am on Monday 9/7}.

\end{document}