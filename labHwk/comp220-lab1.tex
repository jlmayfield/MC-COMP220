\documentclass[10pt]{article}
\usepackage{amsmath}
\usepackage{setspace}
\usepackage{hyperref}

\setlength{\textheight}{9in} \setlength{\topmargin}{-.5in}
\setlength{\textwidth}{6.5in} \setlength{\oddsidemargin}{0in}
\setlength{\evensidemargin}{0in}

\title{COMP220 - Lab 1 \& Homework 1}
\author{ }
\date{Fall 2015}

\begin{document}
\maketitle

The goal of this lab is to get back into C++ and to familiarize yourself with the use of the new IDE, Code::Blocks. To do this you should complete \textit{exercise 7 from chapter 2 of the text} with a few modifications:
\begin{enumerate}
\item Be sure to put your function in a library. 
\item Write tests for your function using the gTest testing framework.
\item Write a main program that provides a basic CLI interface to your function. If your program is called \textit{sqrt} then running \textit{sqrt 4} at the CLI should produce \textit{2}. You may assume valid command line arguments for this assignment;  Error checking and validation of command line arguments is not required but you are strongly encouraged to do it anyway. 
\end{enumerate}

When lab time is over, submit your \textit{source code only} via the \textit{handin} program. The course is, of course, \textit{comp220} and the assignment is \textit{lab1}. Your program does not need to be complete, but it should, as always, compile and run without crashing. For homework, complete the assignment. Submit your homework by class time on Monday 8/31 as assignment \textit{hwk1} via the handin script. 

\end{document}