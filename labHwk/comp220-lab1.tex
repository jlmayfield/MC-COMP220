\documentclass[10pt]{article}
\usepackage{amsmath}
\usepackage{setspace}
\usepackage{hyperref}
\usepackage{booktabs}
\usepackage{listings}
\usepackage{color}

\setlength{\textheight}{9in} \setlength{\topmargin}{-.5in}
\setlength{\textwidth}{6.5in} \setlength{\oddsidemargin}{0in}
\setlength{\evensidemargin}{0in}

\definecolor{dkgreen}{rgb}{0,0.6,0}
\definecolor{gray}{rgb}{0.5,0.5,0.5}
\definecolor{mauve}{rgb}{0.58,0,0.82}

\lstset{
  %frame=tb,
  language=C++,
  aboveskip=3mm,
  belowskip=3mm,
  showstringspaces=false,
  columns=flexible,
  basicstyle={\small\ttfamily},
  numbers=none,
  numberstyle=\tiny\color{gray},
  keywordstyle=\color{blue},
  commentstyle=\color{dkgreen},
  stringstyle=\color{mauve},
  breaklines=true,
  breakatwhitespace=true,
  tabsize=3
}

\title{COMP220 --- Lab 1 \& Homework 1}
\author{ }
\date{Fall 2016}

\begin{document}
\maketitle
\thispagestyle{empty}

\begin{abstract}
  For lab you'll review some basics of complexity and algorithm analysis as well as hone your skills with Logarithmic thinking. For homework you'll dust off your C++ skills and explore the new programming environment: \textit{Code::Blocks}.
\end{abstract}

\section{Lab 1}

\begin{enumerate}
  \item The formula for the conversion from one logarithm base to another tells us that for any two bases $a$ and $b$, $\log_a(n)$ differs from $\log_b(n)$ by exactly a factor of $\frac{1}{\log_b(a)}$.
  \[
  \log_b(n) = \dfrac{\log_a(n)}{\log_a(b)}
  \]
  We're often interested in logs base 2, so let's compute the conversion factor for base $b=2$ for each bases listed in the table below:

  \begin{center}
  \begin{tabular}{lc}
  \underline{Base $a$} & \underline{Factor to convert to $\log_2$} \\
     &  \\
     &  \\
  $e$& \underline{\hspace{1.5in}} \\
     &  \\
     &  \\
  8  & \underline{\hspace{1.5in}} \\
     & \\
     &  \\
  10 & \underline{\hspace{1.5in}} \\
     & \\
     &  \\
  16 & \underline{\hspace{1.5in}} \\
     & \\
     &  \\
  23 & \underline{\hspace{1.5in}} \\
     & \\
     &  \\
  127 & \underline{\hspace{1.5in}} \\
   & \\
   &  \\
  256  & \underline{\hspace{1.5in}} \\
  \end{tabular}
  \end{center}
  What do you notice about converting to $\log_2$ from some base $b=2^k$?

\newpage \thispagestyle{empty}

\item Fill in the following table in three phases. First without using a calculator, fill in the integer lower and upper bound. If the logarithm is an exact integer, then put that as both the lower and upper bound. Then, take a shot a guessing the logarithm's actual value out to 2 decimal places. Finally, use a calculator to compute the actual log to 2 places and fill that in for the final column. When computing actual logarithms, you may only use the log base 10 function on a calculator; use the conversion formula for all other bases.

\begin{center}
\begin{tabular}{lcccc}
\underline{Logarithm} & \underline{Lower Int}. & \underline{Upper Int.} &
  \underline{Guesstimate} & \underline{Actual Value} \\
& & & & \\ & & & & \\
$\log_2(17)$ & \underline{\hspace{1in}} & \underline{\hspace{1in}} & \underline{\hspace{1in}}  & \underline{\hspace{1in}} \\
& & & & \\ & & & & \\
$\log_2(725)$  & \underline{\hspace{1in}} & \underline{\hspace{1in}} & \underline{\hspace{1in}}  & \underline{\hspace{1in}} \\
& & & & \\ & & & & \\
$\log_2(1024)$ & \underline{\hspace{1in}} & \underline{\hspace{1in}} & \underline{\hspace{1in}}  & \underline{\hspace{1in}} \\
 & & & & \\ & & & & \\
$\log_2(7)$ & \underline{\hspace{1in}} & \underline{\hspace{1in}} & \underline{\hspace{1in}}  & \underline{\hspace{1in}} \\
 & & & & \\ & & & & \\
 $\log_8(7)$ & \underline{\hspace{1in}} & \underline{\hspace{1in}} & \underline{\hspace{1in}}  & \underline{\hspace{1in}} \\
 & & & & \\ & & & & \\
$\log_8(178)$ & \underline{\hspace{1in}} & \underline{\hspace{1in}} & \underline{\hspace{1in}}  & \underline{\hspace{1in}} \\
 & & & & \\ & & & & \\
$\log_{16}(25)$ & \underline{\hspace{1in}} & \underline{\hspace{1in}} & \underline{\hspace{1in}}  & \underline{\hspace{1in}} \\
 & & & & \\ & & & & \\
$\log_{16}(333)$ & \underline{\hspace{1in}} & \underline{\hspace{1in}} & \underline{\hspace{1in}}  & \underline{\hspace{1in}} \\
 & & & & \\ & & & & \\
$\log_{10}(15)$ & \underline{\hspace{1in}} & \underline{\hspace{1in}} & \underline{\hspace{1in}}  & \underline{\hspace{1in}} \\
 & & & & \\ & & & & \\
$\log_{10}(150)$ & \underline{\hspace{1in}} & \underline{\hspace{1in}} & \underline{\hspace{1in}}  & \underline{\hspace{1in}} \\
 & & & & \\ & & & & \\
$\log_{10}(145787)$ & \underline{\hspace{1in}} & \underline{\hspace{1in}} & \underline{\hspace{1in}}  & \underline{\hspace{1in}} \\
 & & & & \\ & & & & \\
\end{tabular}
\end{center}

\newpage \thispagestyle{empty}

\item Rank the following complexity from least to greatest order, in terms of resources needs by members of the class, where $1$ is the least resource intense class and numbers proceed up from there.

\begin{center}
\begin{tabular}{ll}
       & \\ & \\
$O(n \log n)$ & \underline{\hspace{1.5in}} \\
       & \\ & \\
$O(n^2)$ & \underline{\hspace{1.5in}} \\
       & \\ & \\
$O(1)$ & \underline{\hspace{1.5in}} \\
       & \\ & \\
$O(2^n)$ & \underline{\hspace{1.5in}} \\
       & \\ & \\
$O(n^3)$ & \underline{\hspace{1.5in}} \\
       & \\ & \\
$O(\log n)$ & \underline{\hspace{1.5in}} \\
       & \\ & \\
$O(n)$ & \underline{\hspace{1.5in}}
\end{tabular}
\end{center}

\item Determine the Big-O for each of the following functions.
\begin{center}
\begin{tabular}{lc}
 & \\ & \\
$\frac{7}{100}n+1.5E16$ & \underline{\hspace{1.5in}} \\
& \\ & \\
$\dfrac{x\log x}{25} - 13x + 5x^3$ & \underline{\hspace{1.5in}} \\
& \\ & \\
$\frac{(n+1)(n+2)}{n}$ & \underline{\hspace{1.5in}} \\
& \\ & \\
$10^9$ & \underline{\hspace{1.5in}} \\
& \\ & \\
$\log y + 135 + \dfrac{y}{2}$ & \underline{\hspace{1.5in}} \\
\end{tabular}
\end{center}

\newpage \thispagestyle{empty}

\item Consider the following loop template containing integer constants $a$ and $s$:
\begin{center}
\begin{lstlisting}
for(int i{a} ; i < n ; i+=s){
}
\end{lstlisting}
\end{center}
\begin{enumerate}
  \item What constraints must be placed on the values of $a$, $s$, and $n$ in order for this loop to terminate properly?
  \vspace{2in}
  \item Assuming the above constraints are met, exactly how many iterations will it perform before it terminates?
  \vspace{2in}
  \item What is the complexity of this loop and what effect, if any, do the values of $a$ and $s$ have on its complexity?
  \newpage \thispagestyle{empty}
\end{enumerate}

\item Consider the following loop template containing integer constants $a$ and $s$:
\begin{center}
\begin{lstlisting}
for(int i{a} ; i > n ; i/=s){
}
\end{lstlisting}
\end{center}
\begin{enumerate}
  \item What constraints must be placed on the values of $a$, $s$, and $n$ in order for this loop to terminate properly?
  \vspace{2in}
  \item Assuming the above constraints are met, exactly how many iterations will it perform before it terminates?
  \vspace{2in}
  \item What is the complexity of this loop and what effect, if any, do the values of $a$ and $s$ have on its complexity?
  \newpage \thispagestyle{empty}
\end{enumerate}

\end{enumerate}
\section{Homework 1}

The goal of this homework is to get back into C++ and to familiarize yourself with the use of the new IDE, Code::Blocks. To do this you should complete \textit{exercise 7 from chapter 2 of the text} with a few modifications:
\begin{enumerate}
\item Be sure to put your function in a library.
\item Write tests for your function using the gTest testing framework.
\item Write a main program that provides a basic CLI interface to your function. If your program is called \textit{sqrt} then running \textit{sqrt 4} at the CLI should produce \textit{2}. You may assume valid command line arguments for this assignment;  Error checking and validation of command line arguments is not required but you are strongly encouraged to do it anyway.
\end{enumerate}

Submit your \textit{source code only} via the \textit{handin} program. The course is, of course, \textit{comp220} and the assignment is \textit{hwk1}. The homework is due prior to next week's lab, so \textbf{9/1 by 2pm}.

\end{document}
