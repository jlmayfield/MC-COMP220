\documentclass[10pt]{article}
\usepackage{amsmath}
\usepackage{setspace}
\usepackage{hyperref}

\setlength{\textheight}{9in} \setlength{\topmargin}{-.5in}
\setlength{\textwidth}{6.5in} \setlength{\oddsidemargin}{0in}
\setlength{\evensidemargin}{0in}

\title{COMP220 --- Lab 6 \& Homework 5}
\author{ }
\date{Fall 2015}

\begin{document}
\maketitle

\begin{abstract}
In this lab you'll tackle a version of the procedure described in problem 21 from chapter 5 of the text. This problem let's you work with some basic map and set methods.
\end{abstract}

\subsection*{The Problem}

Problem 21 in chapter 5 of the text describes an instance of the map inversion problem. The real difficulty here is that where keys  are unique, values need not be.  For example, an area code (key) is attributed to a single state (value) but states can have multiple area codes associated with them.

You need to write an \textit{invert} procedure which takes an integer to string map and produces a string to set of integers map.  In terms of the area code problem this would be taking a map of area codes to states to a state to area codes map.

To practice with mutator design let's envision this procedure as a mutator. You'll give it the map to be inverted by constant reference and a reference to the map that is to be the result. We'll assume the result map is initially empty.

\subsection*{Lab}

You goal in lab is to work on the documentation, a stub, and then focus entirely on tests. If I see a single stitch of implementation without a complete set of tests you can expect to get a 1 for the lab.

\subsection*{Homework}

\begin{center}
  \textbf{Due Monday 10/3} Submit as assignment \textit{hwk5}.
\end{center}

Finish the implementation of \textit{invert}. 

\end{document}
