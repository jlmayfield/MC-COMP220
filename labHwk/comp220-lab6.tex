\documentclass[10pt]{article}
\usepackage{amsmath}
\usepackage{setspace}
\usepackage{hyperref}

\setlength{\textheight}{9in} \setlength{\topmargin}{-.5in}
\setlength{\textwidth}{6.5in} \setlength{\oddsidemargin}{0in}
\setlength{\evensidemargin}{0in}

\title{COMP220 - Lab 6}
\author{ }
\date{Fall 2015}

\begin{document}
\maketitle

\begin{abstract}
For this lab you'll work with the Grid class (pg 210) and practice doing an important traversal pattern in which for each spot in a grid you'll examine all the neighboring locations. This pattern is used in many imagine processing contexts as well as the famous Conway's Game of Life.
\end{abstract}

\section{Problem and Adaptations}

The problem you'll be dealing with is the Minesweeper problem listed in chapter 5 exercise 10. Your solution should:
\begin{itemize}
\item Work on any size grid
\item Utilize a helper procedure named \textit{getCount} (which you must design and write) that is given the grid\textit{mines} (as discussed in the book) and a \textit{row} and \textit{column} index. It then returns the mine count for the location \textit{mines[row][col]}. Note this helper is the core logic to the whole problem. With it written, \textit{fixCounts} just becomes a simple matter of ``visit each spot in \textit{mines}, run \textit{getCounts}, and set the return value to the appropriate location in \textit{counts}''.
\end{itemize}
After you document, declare, stub, and write tests for both procedures. Focus your implementation efforts on the helper. By the end of the day, submit your work as assignment \textit{lab6}. There is no homework, but you're strongly encouraged to finish this assignment up as you can get a lot of mileage from these ``neighborhood'' based traversal patterns.

 

\end{document}